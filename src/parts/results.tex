\chapter{Results}

% header
\rhead{Results}
\lhead{}

% grey line
\vspace{-1.6cm}
\begingroup
\color{gray}
\par\noindent\rule{\textwidth}{0.4pt}
\endgroup


\section{A Results Section}

Some figures are useful to illustrate the results.

% example of a figure
\begin{figure}[h]
    \centering
    \rule{1cm}{1cm} % placeholder for a figure
    \caption{This is a caption.}
    \label{fig:example}
\end{figure}

I can refer to Equation~\ref{eq:example} for the famous equation.

\ldots


\section{Another Results Section}

Some tables are also useful to illustrate other results.

% example of a table
\begin{table}[h]
    \centering
    \caption{This is a different caption. Here we have a standard confusion matrix.}
    \label{tab:example}
    \footnotesize
    \begin{tabular}{@{}cccc@{}}
        \multicolumn{2}{c}{\multirow{2}{*}{}} & \multicolumn{2}{c}{Predicted} \\
        \cmidrule(lr){3-4}
        \multicolumn{2}{c}{} & Positive & Negative \\
        \cmidrule{2-4}
        \multirow[c]{2}{*}{\rotatebox[origin=c]{90}{Real}} & Positive & True Positives & False Negatives \\
        & Negative & False Positives & True Negatives \\
        \cmidrule{2-4}
\end{tabular}
\end{table}
% A good table resource:
% https://www.tablesgenerator.com

\ldots
